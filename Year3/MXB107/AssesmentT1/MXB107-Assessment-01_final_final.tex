% Options for packages loaded elsewhere
\PassOptionsToPackage{unicode}{hyperref}
\PassOptionsToPackage{hyphens}{url}
%
\documentclass[
]{article}
\usepackage{lmodern}
\usepackage{amssymb,amsmath}
\usepackage{ifxetex,ifluatex}
\ifnum 0\ifxetex 1\fi\ifluatex 1\fi=0 % if pdftex
  \usepackage[T1]{fontenc}
  \usepackage[utf8]{inputenc}
  \usepackage{textcomp} % provide euro and other symbols
\else % if luatex or xetex
  \usepackage{unicode-math}
  \defaultfontfeatures{Scale=MatchLowercase}
  \defaultfontfeatures[\rmfamily]{Ligatures=TeX,Scale=1}
\fi
% Use upquote if available, for straight quotes in verbatim environments
\IfFileExists{upquote.sty}{\usepackage{upquote}}{}
\IfFileExists{microtype.sty}{% use microtype if available
  \usepackage[]{microtype}
  \UseMicrotypeSet[protrusion]{basicmath} % disable protrusion for tt fonts
}{}
\makeatletter
\@ifundefined{KOMAClassName}{% if non-KOMA class
  \IfFileExists{parskip.sty}{%
    \usepackage{parskip}
  }{% else
    \setlength{\parindent}{0pt}
    \setlength{\parskip}{6pt plus 2pt minus 1pt}}
}{% if KOMA class
  \KOMAoptions{parskip=half}}
\makeatother
\usepackage{xcolor}
\IfFileExists{xurl.sty}{\usepackage{xurl}}{} % add URL line breaks if available
\IfFileExists{bookmark.sty}{\usepackage{bookmark}}{\usepackage{hyperref}}
\hypersetup{
  pdftitle={MXB107 Assessment 1},
  pdfauthor={Jiyan Zhu},
  hidelinks,
  pdfcreator={LaTeX via pandoc}}
\urlstyle{same} % disable monospaced font for URLs
\usepackage[margin=1in]{geometry}
\usepackage{color}
\usepackage{fancyvrb}
\newcommand{\VerbBar}{|}
\newcommand{\VERB}{\Verb[commandchars=\\\{\}]}
\DefineVerbatimEnvironment{Highlighting}{Verbatim}{commandchars=\\\{\}}
% Add ',fontsize=\small' for more characters per line
\usepackage{framed}
\definecolor{shadecolor}{RGB}{248,248,248}
\newenvironment{Shaded}{\begin{snugshade}}{\end{snugshade}}
\newcommand{\AlertTok}[1]{\textcolor[rgb]{0.94,0.16,0.16}{#1}}
\newcommand{\AnnotationTok}[1]{\textcolor[rgb]{0.56,0.35,0.01}{\textbf{\textit{#1}}}}
\newcommand{\AttributeTok}[1]{\textcolor[rgb]{0.77,0.63,0.00}{#1}}
\newcommand{\BaseNTok}[1]{\textcolor[rgb]{0.00,0.00,0.81}{#1}}
\newcommand{\BuiltInTok}[1]{#1}
\newcommand{\CharTok}[1]{\textcolor[rgb]{0.31,0.60,0.02}{#1}}
\newcommand{\CommentTok}[1]{\textcolor[rgb]{0.56,0.35,0.01}{\textit{#1}}}
\newcommand{\CommentVarTok}[1]{\textcolor[rgb]{0.56,0.35,0.01}{\textbf{\textit{#1}}}}
\newcommand{\ConstantTok}[1]{\textcolor[rgb]{0.00,0.00,0.00}{#1}}
\newcommand{\ControlFlowTok}[1]{\textcolor[rgb]{0.13,0.29,0.53}{\textbf{#1}}}
\newcommand{\DataTypeTok}[1]{\textcolor[rgb]{0.13,0.29,0.53}{#1}}
\newcommand{\DecValTok}[1]{\textcolor[rgb]{0.00,0.00,0.81}{#1}}
\newcommand{\DocumentationTok}[1]{\textcolor[rgb]{0.56,0.35,0.01}{\textbf{\textit{#1}}}}
\newcommand{\ErrorTok}[1]{\textcolor[rgb]{0.64,0.00,0.00}{\textbf{#1}}}
\newcommand{\ExtensionTok}[1]{#1}
\newcommand{\FloatTok}[1]{\textcolor[rgb]{0.00,0.00,0.81}{#1}}
\newcommand{\FunctionTok}[1]{\textcolor[rgb]{0.00,0.00,0.00}{#1}}
\newcommand{\ImportTok}[1]{#1}
\newcommand{\InformationTok}[1]{\textcolor[rgb]{0.56,0.35,0.01}{\textbf{\textit{#1}}}}
\newcommand{\KeywordTok}[1]{\textcolor[rgb]{0.13,0.29,0.53}{\textbf{#1}}}
\newcommand{\NormalTok}[1]{#1}
\newcommand{\OperatorTok}[1]{\textcolor[rgb]{0.81,0.36,0.00}{\textbf{#1}}}
\newcommand{\OtherTok}[1]{\textcolor[rgb]{0.56,0.35,0.01}{#1}}
\newcommand{\PreprocessorTok}[1]{\textcolor[rgb]{0.56,0.35,0.01}{\textit{#1}}}
\newcommand{\RegionMarkerTok}[1]{#1}
\newcommand{\SpecialCharTok}[1]{\textcolor[rgb]{0.00,0.00,0.00}{#1}}
\newcommand{\SpecialStringTok}[1]{\textcolor[rgb]{0.31,0.60,0.02}{#1}}
\newcommand{\StringTok}[1]{\textcolor[rgb]{0.31,0.60,0.02}{#1}}
\newcommand{\VariableTok}[1]{\textcolor[rgb]{0.00,0.00,0.00}{#1}}
\newcommand{\VerbatimStringTok}[1]{\textcolor[rgb]{0.31,0.60,0.02}{#1}}
\newcommand{\WarningTok}[1]{\textcolor[rgb]{0.56,0.35,0.01}{\textbf{\textit{#1}}}}
\usepackage{graphicx,grffile}
\makeatletter
\def\maxwidth{\ifdim\Gin@nat@width>\linewidth\linewidth\else\Gin@nat@width\fi}
\def\maxheight{\ifdim\Gin@nat@height>\textheight\textheight\else\Gin@nat@height\fi}
\makeatother
% Scale images if necessary, so that they will not overflow the page
% margins by default, and it is still possible to overwrite the defaults
% using explicit options in \includegraphics[width, height, ...]{}
\setkeys{Gin}{width=\maxwidth,height=\maxheight,keepaspectratio}
% Set default figure placement to htbp
\makeatletter
\def\fps@figure{htbp}
\makeatother
\setlength{\emergencystretch}{3em} % prevent overfull lines
\providecommand{\tightlist}{%
  \setlength{\itemsep}{0pt}\setlength{\parskip}{0pt}}
\setcounter{secnumdepth}{-\maxdimen} % remove section numbering

\title{MXB107 Assessment 1}
\author{Jiyan Zhu}
\date{Semester 2 2021}

\begin{document}
\maketitle

\textbf{\emph{NOTE THIS ASSESSMENT IS DUE ON 5 September BY 11:59 PM.}}

\textbf{For this Assessment we will use the following dataset:}

\textbf{The dataset} \texttt{episodes} \textbf{included in the MXB107
package for R contains records for 704 episodes of the \emph{Star Trek}
aired between 1966 and 2005. (Type} \texttt{?episodes} \textbf{for a
detailed description of the data.)}

\hypertarget{part-1-summarising-data}{%
\subsection{Part 1: Summarising Data}\label{part-1-summarising-data}}

\hypertarget{question-1}{%
\subsubsection{Question 1}\label{question-1}}

\begin{enumerate}
\def\labelenumi{\alph{enumi}.}
\tightlist
\item
  Name three principles for good practice when creating graphical
  summaries of data.
\end{enumerate}

\textbf{Type your answer here:}\\
1. Having a title is required for graphical summary. The titles should
describe the variables and the relationship correctly in the summary.
Furthermore, if time is the axes, or the data is for a specific period,
those details should be included in the title. 2. Clearly label the axes
and must include should units\\
3. The axes should be match when comparing two sets of data.

\begin{enumerate}
\def\labelenumi{\alph{enumi}.}
\setcounter{enumi}{1}
\tightlist
\item
  Identify three elements of the following graphical summary of data
  that should be corrected.
\end{enumerate}

\begin{center}\includegraphics{MXB107-Assessment-01_final_final_files/figure-latex/unnamed-chunk-2-1} \end{center}

\textbf{Type your answer here:}\\

\begin{enumerate}
\def\labelenumi{\arabic{enumi}.}
\tightlist
\item
  There is no title for the graphical summary.\\
\item
  label is not clear and axes does not include units
\item
  two data sets' axes is not matched.
\end{enumerate}

\begin{enumerate}
\def\labelenumi{\alph{enumi}.}
\setcounter{enumi}{2}
\tightlist
\item
  Create a set of boxplots showing the IMDB rankings for each series of
  \emph{Star Trek}. Discuss the results.
\end{enumerate}

\textbf{Show your code here:}

\begin{Shaded}
\begin{Highlighting}[]
\CommentTok{##  The library MXB107 should be already loaded, if not type:}

\CommentTok{##  library(MXB107)}

\CommentTok{##  If after loading the library if the dataset episodes not available, type:}

\CommentTok{##  data(episodes)}
\KeywordTok{data}\NormalTok{(episodes)}
\KeywordTok{boxplot}\NormalTok{(IMDB.Ranking}\OperatorTok{~}\NormalTok{Series, }\DataTypeTok{data =}\NormalTok{ episodes, }\DataTypeTok{main =} \StringTok{"Compare Rating for all Star Trek series"}\NormalTok{,}
        \DataTypeTok{xlab =} \StringTok{"Series name"}\NormalTok{, }\DataTypeTok{ylab =} \StringTok{"Rating"}\NormalTok{, }\DataTypeTok{col =} \StringTok{"red"}\NormalTok{, }\DataTypeTok{border =} \StringTok{"brown"}\NormalTok{)}
\end{Highlighting}
\end{Shaded}

\begin{center}\includegraphics{MXB107-Assessment-01_final_final_files/figure-latex/unnamed-chunk-3-1} \end{center}

\textbf{Type your answer here:} Box plot shows six things that the
reader can easily pick up, those are The minimum(which is the bottom of
the line), Q1 and Q3(The size of the box from the bottom to the top),
median(the middle line inside the box), maximum (at the top of the
line), and the outlines(circles that is not on the line). overall, the
median for all episodes are about the same, TNG has the highest epsiode
rating, and also the lowest epsiode rating.

\begin{enumerate}
\def\labelenumi{\alph{enumi}.}
\setcounter{enumi}{3}
\tightlist
\item
  Create a pair of histograms comparing the IMDB rankings for episodes
  of \emph{Star Trek: The Next Generation} that pass the Bechdel-Wallace
  Test versus those that failed. Discuss the results.
\end{enumerate}

\textbf{Show your code here:}

\begin{Shaded}
\begin{Highlighting}[]
\CommentTok{##  The library MXB107 should be already loaded, if not type:}

\CommentTok{##  library(MXB107)}

\CommentTok{##  If after loading the library if the dataset episodes not available, type:}

\CommentTok{##  data(episodes)}

 
 \KeywordTok{ggplot}\NormalTok{(episodes }\OperatorTok\StringTok{ }\KeywordTok{filter}\NormalTok{(Series }\OperatorTok{==}\StringTok{ "TNG"}\NormalTok{),}\KeywordTok{aes}\NormalTok{(IMDB.Ranking))}\OperatorTok{+}
\StringTok{  }\KeywordTok{geom_histogram}\NormalTok{(}\KeywordTok{aes}\NormalTok{(}\DataTypeTok{y=}\NormalTok{..density..),}\DataTypeTok{binwidth =} \FloatTok{0.5}\NormalTok{)}\OperatorTok{+}
\StringTok{  }\KeywordTok{facet_wrap}\NormalTok{(}\KeywordTok{vars}\NormalTok{(Bechdel.Wallace.Test))}\OperatorTok{+}
\StringTok{  }\KeywordTok{ylab}\NormalTok{(}\StringTok{"Density"}\NormalTok{)}\OperatorTok{+}
\StringTok{  }\KeywordTok{xlab}\NormalTok{(}\StringTok{"IMDB User Ratings"}\NormalTok{)}\OperatorTok{+}
\StringTok{  }\KeywordTok{ggtitle}\NormalTok{(}\StringTok{"IMDB User Ratings for Star Trek episodes}\CharTok{\textbackslash{}n}\StringTok{ for Passing and Failing the Bechdel-Wallace Test"}\NormalTok{)}\OperatorTok{+}
\StringTok{  }\KeywordTok{theme}\NormalTok{(}\DataTypeTok{plot.title =} \KeywordTok{element_text}\NormalTok{(}\DataTypeTok{hjust=}\FloatTok{0.5}\NormalTok{))}
\end{Highlighting}
\end{Shaded}

\begin{center}\includegraphics{MXB107-Assessment-01_final_final_files/figure-latex/unnamed-chunk-4-1} \end{center}

\textbf{Type your answer here:} I used Density to make sure that both
graphs are in the same scale. There is no major difference between those
two graphs, beside that there is a outliner for the FALSE graph(around
3).

\hypertarget{question-2}{%
\subsubsection{Question 2}\label{question-2}}

\begin{enumerate}
\def\labelenumi{\alph{enumi}.}
\tightlist
\item
  Identify and define three numerical summaries of centrality for data.
\end{enumerate}

\textbf{Type your answer here:}

\begin{enumerate}
\def\labelenumi{\arabic{enumi}.}
\tightlist
\item
  Mean\\
\item
  Median
\item
  Mode
\end{enumerate}

\begin{enumerate}
\def\labelenumi{\alph{enumi}.}
\setcounter{enumi}{1}
\tightlist
\item
  Identify and define three numerical summaries of dispersion for data.
\end{enumerate}

\textbf{Type your answer here:}

\begin{enumerate}
\def\labelenumi{\arabic{enumi}.}
\tightlist
\item
  Range\\
\item
  Variance
\item
  Standard Deviation
\end{enumerate}

\hypertarget{question-3}{%
\subsubsection{Question 3}\label{question-3}}

\begin{enumerate}
\def\labelenumi{\alph{enumi}.}
\tightlist
\item
  For all 704 episodes of \emph{Star Trek} compute the standard
  deviation of their IMDB rankings using the definition of standard
  deviation and then use the empirical rule to estimate the standard
  deviation. Compare and discuss the results.
\end{enumerate}

\textbf{Show your code here:}

\begin{Shaded}
\begin{Highlighting}[]
\CommentTok{##  The library MXB107 should be already loaded, if not type:}

\CommentTok{##  library(MXB107)}

\CommentTok{##  If after loading the library if the dataset episodes not available, type:}

\CommentTok{##  data(episodes)}
\NormalTok{x <-}\StringTok{ }\KeywordTok{c}\NormalTok{(episodes}\OperatorTok{$}\NormalTok{IMDB.Ranking)}
\NormalTok{s <-}\StringTok{ }\KeywordTok{sd}\NormalTok{(x)}

\NormalTok{x1 <-}\StringTok{ }\KeywordTok{range}\NormalTok{(x)}\OperatorTok\KeywordTok{diff}\NormalTok{()}

\NormalTok{shat<-}\StringTok{ }\NormalTok{x1}\OperatorTok{/}\DecValTok{4}

\NormalTok{s}
\end{Highlighting}
\end{Shaded}

\begin{verbatim}
## [1] 0.7760457
\end{verbatim}

\begin{Shaded}
\begin{Highlighting}[]
\NormalTok{shat}
\end{Highlighting}
\end{Shaded}

\begin{verbatim}
## [1] 1.475
\end{verbatim}

\textbf{Type your answer here:} standard deviation is 0.7760457, and
esimate is 1.475 empirical is prediction on standard deviation, and the
reason why the number is larger is due to empirical rule does not count
outlines

\begin{enumerate}
\def\labelenumi{\alph{enumi}.}
\setcounter{enumi}{1}
\tightlist
\item
  For all 704 episodes of \emph{Star Trek} compute the mean and median
  of their IMDB rankings. Do the data appear to be skewed? Compute the
  skew of the data and plot a histogram of the episodes' IMDB rankings,
  do they appear skewed? Compare and discuss the numerical results and
  the your histogram.
\end{enumerate}

\textbf{Show your code here:}

\begin{Shaded}
\begin{Highlighting}[]
\CommentTok{##  The library MXB107 should be already loaded, if not type:}

\CommentTok{##  library(MXB107)}

\CommentTok{##  If after loading the library if the dataset episodes not available, type:}

\CommentTok{##  data(episodes)}
\NormalTok{x <-}\StringTok{ }\KeywordTok{c}\NormalTok{(episodes}\OperatorTok{$}\NormalTok{IMDB.Ranking)}
\NormalTok{num<-(x}\OperatorTok{-}\KeywordTok{mean}\NormalTok{(x))}\OperatorTok{^}\DecValTok{3}\OperatorTok\KeywordTok{mean}\NormalTok{()}
\NormalTok{den<-}\KeywordTok{sd}\NormalTok{(x)}\OperatorTok{^}\DecValTok{3}
\NormalTok{skew<-num}\OperatorTok{/}\NormalTok{den}
\NormalTok{skew}
\end{Highlighting}
\end{Shaded}

\begin{verbatim}
## [1] -0.3873874
\end{verbatim}

\begin{Shaded}
\begin{Highlighting}[]
\KeywordTok{ggplot}\NormalTok{(episodes,  }\KeywordTok{aes}\NormalTok{(}\DataTypeTok{x =}\NormalTok{ IMDB.Ranking))}\OperatorTok{+}
\StringTok{  }\KeywordTok{geom_histogram}\NormalTok{(}\DataTypeTok{binwidth =}\FloatTok{0.1}\NormalTok{,}\KeywordTok{aes}\NormalTok{(}\DataTypeTok{y=}\NormalTok{..density..))}\OperatorTok{+}\KeywordTok{xlab}\NormalTok{(}\StringTok{"IMDB.Ranking"}\NormalTok{)}\OperatorTok{+}\KeywordTok{ggtitle}\NormalTok{(}\StringTok{"IMDB.Ranking"}\NormalTok{)}
\end{Highlighting}
\end{Shaded}

\begin{center}\includegraphics{MXB107-Assessment-01_final_final_files/figure-latex/unnamed-chunk-6-1} \end{center}

:::\{.box\} \textbf{Type your answer here:} skew is -0.3873874, which is
left skew when looking at the graph, the graph is lean to the right
which means left left skew the ower the skew is, the more lean to the
left. which proven that the answer is correct

\hypertarget{part-2-computing-basic-probabilities-for-events}{%
\subsection{Part 2: Computing Basic Probabilities for
Events}\label{part-2-computing-basic-probabilities-for-events}}

\hypertarget{question-1-1}{%
\subsubsection{Question 1}\label{question-1-1}}

\begin{enumerate}
\def\labelenumi{\alph{enumi}.}
\tightlist
\item
  What is the classical definition of probability?
\end{enumerate}

\textbf{Type your answer here:} The Definition of probability is that
the probability is a mathematical tool used for quantifying uncertainty.
Probability provides tools to compute the likely hood for both simple
events and compound events. Furthermore, it also provides the base for
modeling random variable.

\begin{enumerate}
\def\labelenumi{\alph{enumi}.}
\setcounter{enumi}{1}
\tightlist
\item
  What is the probability that a randomly selected episode of \emph{Star
  Trek} will pass the Bechdel-Wallace Test?
\end{enumerate}

\textbf{Show your code here:}

\begin{Shaded}
\begin{Highlighting}[]
\CommentTok{##  The library MXB107 should be already loaded, if not type:}

\CommentTok{##  library(MXB107)}

\CommentTok{##  If after loading the library if the dataset episodes not available, type:}

\CommentTok{##  data(episodes)}

\KeywordTok{table}\NormalTok{(episodes}\OperatorTok{$}\NormalTok{Bechdel.Wallace.Test)}
\end{Highlighting}
\end{Shaded}

\begin{verbatim}
## 
## FALSE  TRUE 
##   338   366
\end{verbatim}

\begin{Shaded}
\begin{Highlighting}[]
\NormalTok{Num_way_A =}\StringTok{ }\DecValTok{366}
\NormalTok{total =}\StringTok{ }\DecValTok{704}
\NormalTok{answer =}\StringTok{ }\NormalTok{Num_way_A}\OperatorTok{/}\NormalTok{total}
\NormalTok{answer}
\end{Highlighting}
\end{Shaded}

\begin{verbatim}
## [1] 0.5198864
\end{verbatim}

\textbf{Type your answer here:} For this question we can use probability
rules, where Pr(A) = number of ways that A can occur, over total number
of outcomes. Pr(A) = 366/704 = 0.5198864.

\hypertarget{question-2-1}{%
\subsubsection{Question 2}\label{question-2-1}}

\begin{enumerate}
\def\labelenumi{\alph{enumi}.}
\tightlist
\item
  What is the definition of joint probability?
\end{enumerate}

\textbf{Type your answer here:} \[
P(A \cap B) = Pr(A) * Pr(B)
\]

\begin{enumerate}
\def\labelenumi{\alph{enumi}.}
\setcounter{enumi}{1}
\tightlist
\item
  What is the probability that an original series episode passes the
  Bechdel-Wallace Test?
\end{enumerate}

\textbf{Show your code here:}

\begin{Shaded}
\begin{Highlighting}[]
\CommentTok{##  The library MXB107 should be already loaded, if not type:}

\CommentTok{##  library(MXB107)}

\CommentTok{##  If after loading the library if the dataset episodes not available, type:}

\CommentTok{##  data(episodes)}

\NormalTok{episodes }\OperatorTok\StringTok{ }\KeywordTok{filter}\NormalTok{(Series }\OperatorTok{==}\StringTok{"TOS"}\NormalTok{)}
\end{Highlighting}
\end{Shaded}

\begin{verbatim}
## # A tibble: 80 x 57
##    Series Series.Name Season Episode IMDB.Ranking Title Star.date Air.date
##    <chr>  <chr>        <dbl>   <dbl>        <dbl> <chr> <chr>     <chr>   
##  1 TOS    The Origin~      1       1          7.3 The ~ 1513.1    8/9/66  
##  2 TOS    The Origin~      1       2          7.2 Char~ 1533.6    15/9/66 
##  3 TOS    The Origin~      1       3          7.8 Wher~ 1312.4    22/9/66 
##  4 TOS    The Origin~      1       4          8   The ~ 1704.2    29/9/66 
##  5 TOS    The Origin~      1       5          7.8 The ~ 1672.1    6/10/66 
##  6 TOS    The Origin~      1       6          6.9 Mudd~ 1329.8    13/10/66
##  7 TOS    The Origin~      1       7          7.6 What~ 2712.4    20/10/66
##  8 TOS    The Origin~      1       8          7.1 Miri  2713.5    27/10/66
##  9 TOS    The Origin~      1       9          7.5 Dagg~ 2715.1    3/11/66 
## 10 TOS    The Origin~      1      10          8.2 The ~ 1512.2    10/11/66
## # ... with 70 more rows, and 49 more variables: Bechdel.Wallace.Test <lgl>,
## #   Director <chr>, Writer.1 <chr>, Writer.2 <chr>, Writer.3 <chr>,
## #   Writer.4 <chr>, Writer.5 <chr>, Writer.6 <chr>, Female.Director <lgl>,
## #   Female.Writer.1 <lgl>, Female.Writer.2 <lgl>, Female.Writer.3 <lgl>,
## #   Female.Writer.4 <lgl>, Female.Writer.5 <lgl>, Female.Writer.6 <lgl>,
## #   Executive.Producer.1 <chr>, Executive.Producer.2 <chr>,
## #   Executive.Producer.3 <chr>, Co.Executive.Producer.1 <chr>,
## #   Co.Executive.Producer.2 <chr>, Co.Executive.Producer.3 <chr>,
## #   Producer.1 <chr>, Producer.2 <chr>, Producer.3 <chr>, Producer.4 <chr>,
## #   Co.Producer.1 <chr>, Co.Producer.2 <chr>, Co.Producer.3 <chr>,
## #   Co.Producer.4 <chr>, Co.Producer.5 <chr>, Associate.Producer.1 <chr>,
## #   Associate.Producer.2 <chr>, Supervising.Producer.1 <chr>,
## #   Supervising.Producer.2 <chr>, Supervising.Producer.3 <chr>,
## #   Co.Supervising.Producer.1 <chr>, Co.Supervising.Producer.2 <chr>,
## #   Line.Producer <chr>, Coordinating.Producer <chr>,
## #   Consulting.Producer.1 <chr>, Consulting.Producer.2 <chr>,
## #   Female.Executive.Producer <lgl>, Female.Co.Executive.Producer <lgl>,
## #   Female.Producer <lgl>, Female.Co.Producer <lgl>,
## #   Female.Associate.Producer <lgl>, Female.Supervising.Producer <lgl>,
## #   Female.Co.Supervising.Producer <lgl>, Female.Line.Producer <lgl>
\end{verbatim}

\begin{Shaded}
\begin{Highlighting}[]
\NormalTok{episodes[}\KeywordTok{which}\NormalTok{(episodes}\OperatorTok{$}\NormalTok{Bechdel.Wallace.Test }\OperatorTok{==}\StringTok{ }\OtherTok{TRUE} \OperatorTok{&}\StringTok{ }\NormalTok{episodes}\OperatorTok{$}\NormalTok{Series}\OperatorTok{==}\StringTok{'TOS'}\NormalTok{),]}
\end{Highlighting}
\end{Shaded}

\begin{verbatim}
## # A tibble: 5 x 57
##   Series Series.Name Season Episode IMDB.Ranking Title Star.date Air.date
##   <chr>  <chr>        <dbl>   <dbl>        <dbl> <chr> <chr>     <chr>   
## 1 TOS    The Origin~      2       3          7.8 The ~ 3541.9    29/9/67 
## 2 TOS    The Origin~      2       8          7.5 I, M~ 4513.3    3/11/67 
## 3 TOS    The Origin~      3      18          6.3 The ~ 5725.3    31/1/69 
## 4 TOS    The Origin~      3      21          7.2 The ~ 5818.4    28/2/69 
## 5 TOS    The Origin~      3      24          7   Turn~ 5928.5    3/6/69  
## # ... with 49 more variables: Bechdel.Wallace.Test <lgl>, Director <chr>,
## #   Writer.1 <chr>, Writer.2 <chr>, Writer.3 <chr>, Writer.4 <chr>,
## #   Writer.5 <chr>, Writer.6 <chr>, Female.Director <lgl>,
## #   Female.Writer.1 <lgl>, Female.Writer.2 <lgl>, Female.Writer.3 <lgl>,
## #   Female.Writer.4 <lgl>, Female.Writer.5 <lgl>, Female.Writer.6 <lgl>,
## #   Executive.Producer.1 <chr>, Executive.Producer.2 <chr>,
## #   Executive.Producer.3 <chr>, Co.Executive.Producer.1 <chr>,
## #   Co.Executive.Producer.2 <chr>, Co.Executive.Producer.3 <chr>,
## #   Producer.1 <chr>, Producer.2 <chr>, Producer.3 <chr>, Producer.4 <chr>,
## #   Co.Producer.1 <chr>, Co.Producer.2 <chr>, Co.Producer.3 <chr>,
## #   Co.Producer.4 <chr>, Co.Producer.5 <chr>, Associate.Producer.1 <chr>,
## #   Associate.Producer.2 <chr>, Supervising.Producer.1 <chr>,
## #   Supervising.Producer.2 <chr>, Supervising.Producer.3 <chr>,
## #   Co.Supervising.Producer.1 <chr>, Co.Supervising.Producer.2 <chr>,
## #   Line.Producer <chr>, Coordinating.Producer <chr>,
## #   Consulting.Producer.1 <chr>, Consulting.Producer.2 <chr>,
## #   Female.Executive.Producer <lgl>, Female.Co.Executive.Producer <lgl>,
## #   Female.Producer <lgl>, Female.Co.Producer <lgl>,
## #   Female.Associate.Producer <lgl>, Female.Supervising.Producer <lgl>,
## #   Female.Co.Supervising.Producer <lgl>, Female.Line.Producer <lgl>
\end{verbatim}

\begin{Shaded}
\begin{Highlighting}[]
\NormalTok{num_pass =}\StringTok{ }\DecValTok{5}
\NormalTok{total =}\StringTok{ }\DecValTok{80}

\NormalTok{answer =}\StringTok{ }\NormalTok{num_pass }\OperatorTok{/}\StringTok{ }\NormalTok{total}
\NormalTok{answer}
\end{Highlighting}
\end{Shaded}

\begin{verbatim}
## [1] 0.0625
\end{verbatim}

\textbf{Type your answer here:} same as the previous question, total
number of appear over number of items. which is 0.0625

\hypertarget{question-3-1}{%
\subsubsection{Question 3}\label{question-3-1}}

\begin{enumerate}
\def\labelenumi{\alph{enumi}.}
\tightlist
\item
  What is the definition of conditional probability?
\item
  What is the probability that an episode fails the Bechdel-Wallace Test
  given that it is an episode from \emph{Star Trek: Deep Space Nine}?
\end{enumerate}

\textbf{Show your code here:}

\begin{Shaded}
\begin{Highlighting}[]
\CommentTok{##  The library MXB107 should be already loaded, if not type:}

\CommentTok{##  library(MXB107)}

\CommentTok{##  If after loading the library if the dataset episodes not available, type:}

\CommentTok{##  data(episodes)}
\NormalTok{episodes }\OperatorTok\StringTok{ }\KeywordTok{filter}\NormalTok{(Bechdel.Wallace.Test }\OperatorTok{==}\StringTok{ }\OtherTok{FALSE}\NormalTok{, Series }\OperatorTok{==}\StringTok{"DS9"}\NormalTok{)}
\end{Highlighting}
\end{Shaded}

\begin{verbatim}
## # A tibble: 76 x 57
##    Series Series.Name Season Episode IMDB.Ranking Title Star.date Air.date
##    <chr>  <chr>        <dbl>   <dbl>        <dbl> <chr> <chr>     <chr>   
##  1 DS9    Deep Space~      1       1          7.5 Emis~ 46379.1   3/1/93  
##  2 DS9    Deep Space~      1       3          7.1 Past~ Unknown   10/1/93 
##  3 DS9    Deep Space~      1       5          7   Babel 46423.7   24/1/93 
##  4 DS9    Deep Space~      1       6          7.7 Capt~ Unknown   31/1/93 
##  5 DS9    Deep Space~      1      11          7.2 The ~ Unknown   21/3/93 
##  6 DS9    Deep Space~      1      12          7.1 Vort~ Unknown   18/4/93 
##  7 DS9    Deep Space~      1      14          6.1 The ~ 46729.1   2/5/93  
##  8 DS9    Deep Space~      2       1          7.8 The ~ Unknown   26/9/93 
##  9 DS9    Deep Space~      2       5          7.6 Card~ 47177.2   24/10/93
## 10 DS9    Deep Space~      2       9          6.3 Seco~ 47329.4   21/11/93
## # ... with 66 more rows, and 49 more variables: Bechdel.Wallace.Test <lgl>,
## #   Director <chr>, Writer.1 <chr>, Writer.2 <chr>, Writer.3 <chr>,
## #   Writer.4 <chr>, Writer.5 <chr>, Writer.6 <chr>, Female.Director <lgl>,
## #   Female.Writer.1 <lgl>, Female.Writer.2 <lgl>, Female.Writer.3 <lgl>,
## #   Female.Writer.4 <lgl>, Female.Writer.5 <lgl>, Female.Writer.6 <lgl>,
## #   Executive.Producer.1 <chr>, Executive.Producer.2 <chr>,
## #   Executive.Producer.3 <chr>, Co.Executive.Producer.1 <chr>,
## #   Co.Executive.Producer.2 <chr>, Co.Executive.Producer.3 <chr>,
## #   Producer.1 <chr>, Producer.2 <chr>, Producer.3 <chr>, Producer.4 <chr>,
## #   Co.Producer.1 <chr>, Co.Producer.2 <chr>, Co.Producer.3 <chr>,
## #   Co.Producer.4 <chr>, Co.Producer.5 <chr>, Associate.Producer.1 <chr>,
## #   Associate.Producer.2 <chr>, Supervising.Producer.1 <chr>,
## #   Supervising.Producer.2 <chr>, Supervising.Producer.3 <chr>,
## #   Co.Supervising.Producer.1 <chr>, Co.Supervising.Producer.2 <chr>,
## #   Line.Producer <chr>, Coordinating.Producer <chr>,
## #   Consulting.Producer.1 <chr>, Consulting.Producer.2 <chr>,
## #   Female.Executive.Producer <lgl>, Female.Co.Executive.Producer <lgl>,
## #   Female.Producer <lgl>, Female.Co.Producer <lgl>,
## #   Female.Associate.Producer <lgl>, Female.Supervising.Producer <lgl>,
## #   Female.Co.Supervising.Producer <lgl>, Female.Line.Producer <lgl>
\end{verbatim}

\begin{Shaded}
\begin{Highlighting}[]
\KeywordTok{table}\NormalTok{(episodes}\OperatorTok{$}\NormalTok{Series)}
\end{Highlighting}
\end{Shaded}

\begin{verbatim}
## 
## DS9 ENT TNG TOS VOY 
## 176  98 178  80 172
\end{verbatim}

\begin{Shaded}
\begin{Highlighting}[]
\NormalTok{pAandB =}\StringTok{ }\DecValTok{76}\OperatorTok{/}\DecValTok{704}
\NormalTok{pB =}\StringTok{ }\DecValTok{176}\OperatorTok{/}\DecValTok{704}
\NormalTok{pAandB}\OperatorTok{/}\NormalTok{pB}
\end{Highlighting}
\end{Shaded}

\begin{verbatim}
## [1] 0.4318182
\end{verbatim}

\textbf{Type your answer here:} a) The definition of conditional
probability is P(B\textbar A) = P(A∩B) / P(A) b) We can apply
conditional probability. P(B\textbar A) = P(A∩B) / P(A) where P(A∩B) =
76/704, and P(a) is 176/704 answer is 0.4318182

\hypertarget{question-4}{%
\subsubsection{Question 4}\label{question-4}}

\begin{enumerate}
\def\labelenumi{\alph{enumi}.}
\tightlist
\item
  What is Bayes' Theorem
\end{enumerate}

\textbf{Type your answer here:} \[
Pr(B|A) = Pr(A∩B)Pr(B)/Pr(A)
\]

\begin{enumerate}
\def\labelenumi{\alph{enumi}.}
\setcounter{enumi}{1}
\tightlist
\item
  Given that an episode passes the Bechdel-Wallace Test what is the
  probability that is was from Season 3 of \emph{Star Trek: Voyager}
\end{enumerate}

\textbf{Show your code here:}

\begin{Shaded}
\begin{Highlighting}[]
\CommentTok{##  The library MXB107 should be already loaded, if not type:}

\CommentTok{##  library(MXB107)}

\CommentTok{##  If after loading the library if the dataset episodes not available, type:}

\CommentTok{##  data(episodes)}

\NormalTok{episodes }\OperatorTok\StringTok{ }\KeywordTok{filter}\NormalTok{( Series }\OperatorTok{==}\StringTok{"VOY"}\NormalTok{)}
\end{Highlighting}
\end{Shaded}

\begin{verbatim}
## # A tibble: 172 x 57
##    Series Series.Name Season Episode IMDB.Ranking Title Star.date Air.date
##    <chr>  <chr>        <dbl>   <dbl>        <dbl> <chr> <chr>     <chr>   
##  1 VOY    Voyager          1       1          7.4 Care~ 48315.6   16/1/95 
##  2 VOY    Voyager          1       2          7.4 Care~ 48315.6   16/1/95 
##  3 VOY    Voyager          1       3          7.3 Para~ 48439.7   23/1/95 
##  4 VOY    Voyager          1       4          7.2 Time~ Unknown   30/1/95 
##  5 VOY    Voyager          1       5          7.2 Phage 48532.4   6/2/95  
##  6 VOY    Voyager          1       6          6.6 The ~ 48546.2   13/2/95 
##  7 VOY    Voyager          1       7          8.3 Eye ~ 48579.4   20/2/95 
##  8 VOY    Voyager          1       8          6.7 Ex P~ Unknown   27/2/95 
##  9 VOY    Voyager          1       9          6.8 Eman~ 48623.5   13/3/95 
## 10 VOY    Voyager          1      10          7.4 Prim~ 48642.5   20/3/95 
## # ... with 162 more rows, and 49 more variables: Bechdel.Wallace.Test <lgl>,
## #   Director <chr>, Writer.1 <chr>, Writer.2 <chr>, Writer.3 <chr>,
## #   Writer.4 <chr>, Writer.5 <chr>, Writer.6 <chr>, Female.Director <lgl>,
## #   Female.Writer.1 <lgl>, Female.Writer.2 <lgl>, Female.Writer.3 <lgl>,
## #   Female.Writer.4 <lgl>, Female.Writer.5 <lgl>, Female.Writer.6 <lgl>,
## #   Executive.Producer.1 <chr>, Executive.Producer.2 <chr>,
## #   Executive.Producer.3 <chr>, Co.Executive.Producer.1 <chr>,
## #   Co.Executive.Producer.2 <chr>, Co.Executive.Producer.3 <chr>,
## #   Producer.1 <chr>, Producer.2 <chr>, Producer.3 <chr>, Producer.4 <chr>,
## #   Co.Producer.1 <chr>, Co.Producer.2 <chr>, Co.Producer.3 <chr>,
## #   Co.Producer.4 <chr>, Co.Producer.5 <chr>, Associate.Producer.1 <chr>,
## #   Associate.Producer.2 <chr>, Supervising.Producer.1 <chr>,
## #   Supervising.Producer.2 <chr>, Supervising.Producer.3 <chr>,
## #   Co.Supervising.Producer.1 <chr>, Co.Supervising.Producer.2 <chr>,
## #   Line.Producer <chr>, Coordinating.Producer <chr>,
## #   Consulting.Producer.1 <chr>, Consulting.Producer.2 <chr>,
## #   Female.Executive.Producer <lgl>, Female.Co.Executive.Producer <lgl>,
## #   Female.Producer <lgl>, Female.Co.Producer <lgl>,
## #   Female.Associate.Producer <lgl>, Female.Supervising.Producer <lgl>,
## #   Female.Co.Supervising.Producer <lgl>, Female.Line.Producer <lgl>
\end{verbatim}

\begin{Shaded}
\begin{Highlighting}[]
\NormalTok{episodes }\OperatorTok\StringTok{ }\KeywordTok{filter}\NormalTok{( Series }\OperatorTok{==}\StringTok{"VOY"}\NormalTok{, Season }\OperatorTok{==}\StringTok{ '3'}\NormalTok{, Bechdel.Wallace.Test}\OperatorTok{==}\StringTok{"TRUE"}\NormalTok{)}
\end{Highlighting}
\end{Shaded}

\begin{verbatim}
## # A tibble: 17 x 57
##    Series Series.Name Season Episode IMDB.Ranking Title Star.date Air.date
##    <chr>  <chr>        <dbl>   <dbl>        <dbl> <chr> <chr>     <chr>   
##  1 VOY    Voyager          3       1          8   Basi~ 50023.4   4/9/96  
##  2 VOY    Voyager          3       3          6.9 The ~ 50156.2   18/9/96 
##  3 VOY    Voyager          3       4          7.3 The ~ 50252.3   25/9/96 
##  4 VOY    Voyager          3       5          6.8 Fals~ 50074.3   2/10/96 
##  5 VOY    Voyager          3       6          7.2 Reme~ 50203.1   9/10/96 
##  6 VOY    Voyager          3       8          8.4 Futu~ Unknown   6/11/96 
##  7 VOY    Voyager          3      10          6.8 Warl~ 50348.1   20/11/96
##  8 VOY    Voyager          3      11          7.4 The ~ 50384.2   27/11/96
##  9 VOY    Voyager          3      14          6.9 Alte~ 50460.3   15/1/97 
## 10 VOY    Voyager          3      16          7.4 Bloo~ 50537.2   5/2/97  
## 11 VOY    Voyager          3      17          7.9 Unity 50614.2   12/2/97 
## 12 VOY    Voyager          3      20          6.2 Favo~ 50732.4   19/3/97 
## 13 VOY    Voyager          3      21          7.9 Befo~ Unknown   9/4/97  
## 14 VOY    Voyager          3      22          7.5 Real~ 50836.2   23/4/97 
## 15 VOY    Voyager          3      24          7.6 Disp~ 50912.4   7/5/97  
## 16 VOY    Voyager          3      25          8.2 Wors~ 50953.4   14/5/97 
## 17 VOY    Voyager          3      26          8.9 Scor~ 50984.3   21/5/97 
## # ... with 49 more variables: Bechdel.Wallace.Test <lgl>, Director <chr>,
## #   Writer.1 <chr>, Writer.2 <chr>, Writer.3 <chr>, Writer.4 <chr>,
## #   Writer.5 <chr>, Writer.6 <chr>, Female.Director <lgl>,
## #   Female.Writer.1 <lgl>, Female.Writer.2 <lgl>, Female.Writer.3 <lgl>,
## #   Female.Writer.4 <lgl>, Female.Writer.5 <lgl>, Female.Writer.6 <lgl>,
## #   Executive.Producer.1 <chr>, Executive.Producer.2 <chr>,
## #   Executive.Producer.3 <chr>, Co.Executive.Producer.1 <chr>,
## #   Co.Executive.Producer.2 <chr>, Co.Executive.Producer.3 <chr>,
## #   Producer.1 <chr>, Producer.2 <chr>, Producer.3 <chr>, Producer.4 <chr>,
## #   Co.Producer.1 <chr>, Co.Producer.2 <chr>, Co.Producer.3 <chr>,
## #   Co.Producer.4 <chr>, Co.Producer.5 <chr>, Associate.Producer.1 <chr>,
## #   Associate.Producer.2 <chr>, Supervising.Producer.1 <chr>,
## #   Supervising.Producer.2 <chr>, Supervising.Producer.3 <chr>,
## #   Co.Supervising.Producer.1 <chr>, Co.Supervising.Producer.2 <chr>,
## #   Line.Producer <chr>, Coordinating.Producer <chr>,
## #   Consulting.Producer.1 <chr>, Consulting.Producer.2 <chr>,
## #   Female.Executive.Producer <lgl>, Female.Co.Executive.Producer <lgl>,
## #   Female.Producer <lgl>, Female.Co.Producer <lgl>,
## #   Female.Associate.Producer <lgl>, Female.Supervising.Producer <lgl>,
## #   Female.Co.Supervising.Producer <lgl>, Female.Line.Producer <lgl>
\end{verbatim}

\begin{Shaded}
\begin{Highlighting}[]
\NormalTok{t1 =}\StringTok{ }\DecValTok{17}\OperatorTok{/}\DecValTok{704}
\NormalTok{t2 =}\StringTok{ }\DecValTok{366}\OperatorTok{/}\DecValTok{704}
\NormalTok{t1}\OperatorTok{/}\NormalTok{t2}
\end{Highlighting}
\end{Shaded}

\begin{verbatim}
## [1] 0.04644809
\end{verbatim}

\begin{Shaded}
\begin{Highlighting}[]
\NormalTok{episodes }\OperatorTok\StringTok{ }\KeywordTok{filter}\NormalTok{( Series }\OperatorTok{==}\StringTok{"VOY"}\NormalTok{, Season }\OperatorTok{==}\StringTok{ '3'}\NormalTok{)}
\end{Highlighting}
\end{Shaded}

\begin{verbatim}
## # A tibble: 26 x 57
##    Series Series.Name Season Episode IMDB.Ranking Title Star.date Air.date
##    <chr>  <chr>        <dbl>   <dbl>        <dbl> <chr> <chr>     <chr>   
##  1 VOY    Voyager          3       1          8   Basi~ 50023.4   4/9/96  
##  2 VOY    Voyager          3       2          7.9 Flas~ 50126.4   11/9/96 
##  3 VOY    Voyager          3       3          6.9 The ~ 50156.2   18/9/96 
##  4 VOY    Voyager          3       4          7.3 The ~ 50252.3   25/9/96 
##  5 VOY    Voyager          3       5          6.8 Fals~ 50074.3   2/10/96 
##  6 VOY    Voyager          3       6          7.2 Reme~ 50203.1   9/10/96 
##  7 VOY    Voyager          3       7          6   Sacr~ 50063.2   30/10/96
##  8 VOY    Voyager          3       8          8.4 Futu~ Unknown   6/11/96 
##  9 VOY    Voyager          3       9          8.3 Futu~ 50312.6   13/11/96
## 10 VOY    Voyager          3      10          6.8 Warl~ 50348.1   20/11/96
## # ... with 16 more rows, and 49 more variables: Bechdel.Wallace.Test <lgl>,
## #   Director <chr>, Writer.1 <chr>, Writer.2 <chr>, Writer.3 <chr>,
## #   Writer.4 <chr>, Writer.5 <chr>, Writer.6 <chr>, Female.Director <lgl>,
## #   Female.Writer.1 <lgl>, Female.Writer.2 <lgl>, Female.Writer.3 <lgl>,
## #   Female.Writer.4 <lgl>, Female.Writer.5 <lgl>, Female.Writer.6 <lgl>,
## #   Executive.Producer.1 <chr>, Executive.Producer.2 <chr>,
## #   Executive.Producer.3 <chr>, Co.Executive.Producer.1 <chr>,
## #   Co.Executive.Producer.2 <chr>, Co.Executive.Producer.3 <chr>,
## #   Producer.1 <chr>, Producer.2 <chr>, Producer.3 <chr>, Producer.4 <chr>,
## #   Co.Producer.1 <chr>, Co.Producer.2 <chr>, Co.Producer.3 <chr>,
## #   Co.Producer.4 <chr>, Co.Producer.5 <chr>, Associate.Producer.1 <chr>,
## #   Associate.Producer.2 <chr>, Supervising.Producer.1 <chr>,
## #   Supervising.Producer.2 <chr>, Supervising.Producer.3 <chr>,
## #   Co.Supervising.Producer.1 <chr>, Co.Supervising.Producer.2 <chr>,
## #   Line.Producer <chr>, Coordinating.Producer <chr>,
## #   Consulting.Producer.1 <chr>, Consulting.Producer.2 <chr>,
## #   Female.Executive.Producer <lgl>, Female.Co.Executive.Producer <lgl>,
## #   Female.Producer <lgl>, Female.Co.Producer <lgl>,
## #   Female.Associate.Producer <lgl>, Female.Supervising.Producer <lgl>,
## #   Female.Co.Supervising.Producer <lgl>, Female.Line.Producer <lgl>
\end{verbatim}

\textbf{Type your answer here:} Same as above question 0.046

\begin{enumerate}
\def\labelenumi{\alph{enumi}.}
\setcounter{enumi}{2}
\tightlist
\item
  Is this probability greater or less than the marginal probability that
  a randomly selected episode is from Season 3 of \emph{Star Trek:
  Voyager}? Why?
\end{enumerate}

\textbf{Type your answer here:} 0.036, which is more likely. This is
because the the passing rate on voyager season 3 is high, which result
in higher chance of picking an episode that is passed from season 3
voyager.

\hypertarget{part-3-modelling-with-probability-distributions}{%
\subsection{Part 3: Modelling with Probability
Distributions}\label{part-3-modelling-with-probability-distributions}}

\hypertarget{question-1-2}{%
\subsubsection{Question 1}\label{question-1-2}}

\begin{enumerate}
\def\labelenumi{\alph{enumi}.}
\tightlist
\item
  Define a Bernoulli random variable.
\end{enumerate}

\textbf{Type your answer here:} Bernoulli random trail is a experiment
with dichotomous outcome as well as single probability assigned to be a
positive outcome

\begin{enumerate}
\def\labelenumi{\alph{enumi}.}
\setcounter{enumi}{1}
\tightlist
\item
  Assume I have a fair coin, What is the probability that I will need
  more than two coin tosses to get a ``heads''?
\end{enumerate}

\textbf{Show your code here:}

\begin{Shaded}
\begin{Highlighting}[]
\CommentTok{##  The library MXB107 should be already loaded, if not type:}

\CommentTok{##  library(MXB107)}

\CommentTok{##  If after loading the library if the dataset episodes not available, type:}

\CommentTok{##  data(episodes)}
\DecValTok{1} \OperatorTok{-}\StringTok{ }\KeywordTok{sum}\NormalTok{(}\KeywordTok{dbinom}\NormalTok{(}\DecValTok{1}\OperatorTok{:}\DecValTok{2}\NormalTok{, }\DecValTok{2}\NormalTok{, }\FloatTok{0.5}\NormalTok{))}
\end{Highlighting}
\end{Shaded}

\begin{verbatim}
## [1] 0.25
\end{verbatim}

\textbf{Type your answer here:} we can apply the binomial probability
distribution here. where to get the probability that is more than two we
can subtract the probability that is less that two. the answer is 0.25.

\begin{enumerate}
\def\labelenumi{\alph{enumi}.}
\setcounter{enumi}{2}
\tightlist
\item
  Define a geometrically distributed random variable and Write out the
  probability mass distribution for a geometric probability
  distribution. Define the process that gives rise to a geometrically
  distributed random variable in terms of Bernoulli trials.
\end{enumerate}

\textbf{Show your code here:}

\begin{Shaded}
\begin{Highlighting}[]
\CommentTok{##  The library MXB107 should be already loaded, if not type:}

\CommentTok{##  library(MXB107)}

\CommentTok{##  If after loading the library if the dataset episodes not available, type:}

\CommentTok{##  data(episodes)}
\end{Highlighting}
\end{Shaded}

\textbf{Type your answer here:} Geometric distribution is the
probability distribution of number of failures from repeating bernoulli,
untill the first success Let X be the number of trails until success.
Probability mass function for X is P(X=x) = {[}(1-p)\^{}x-1{]}*p X =
1,2,3\ldots{} X follows geometric distribution.

\begin{enumerate}
\def\labelenumi{\alph{enumi}.}
\setcounter{enumi}{3}
\tightlist
\item
  If the overall proportion of \emph{Star Trek} episodes that pass the
  Bechdel-Wallace Test is \(0.52\) then assume I begin watching episodes
  selecting them at random, how many episodes do I have to watch until
  the probability I see at least on episode that passes the
  Bechdel-Wallace Test is more than 95\%?
\end{enumerate}

\textbf{Show your code here:}

\begin{Shaded}
\begin{Highlighting}[]
\CommentTok{##  The library MXB107 should be already loaded, if not type:}

\CommentTok{##  library(MXB107)}

\CommentTok{##  If after loading the library if the dataset episodes not available, type:}

\CommentTok{##  data(episodes)}
\NormalTok{answer =}\StringTok{ }\KeywordTok{pgeom}\NormalTok{(}\DataTypeTok{q =} \DecValTok{0}\OperatorTok{:}\DecValTok{10}\NormalTok{, }\DataTypeTok{prob =} \FloatTok{0.52}\NormalTok{, }\DataTypeTok{lower.tail =} \OtherTok{TRUE}\NormalTok{)}
\NormalTok{answer }
\end{Highlighting}
\end{Shaded}

\begin{verbatim}
##  [1] 0.5200000 0.7696000 0.8894080 0.9469158 0.9745196 0.9877694 0.9941293
##  [8] 0.9971821 0.9986474 0.9993507 0.9996884
\end{verbatim}

\textbf{Type your answer here:} by viewing the answer by using CDF, it
is the number 5 position, but the question is asking until which means
-1 answer is 4.

\hypertarget{question-2-2}{%
\subsubsection{Question 2}\label{question-2-2}}

\begin{enumerate}
\def\labelenumi{\alph{enumi}.}
\tightlist
\item
  I have a coin that comes up heads for a given coin toss with
  probability \(p\). If I toss the coin \(n\) times, on average how many
  heads should I get? What is the standard deviation for the random
  variable \(X=\) number of heads in \(n\) coin tosses?
\end{enumerate}

\textbf{Type your answer here:} E(X)=np and Var(X)=np(1−p).

\begin{enumerate}
\def\labelenumi{\alph{enumi}.}
\setcounter{enumi}{1}
\tightlist
\item
  Describe the a binomial random variable in terms of Bernoulli trials.
  For what value of \(p\) is the variance for a binomial random variable
  maximised?
\end{enumerate}

\textbf{Type your answer here:} p=0.5

\begin{enumerate}
\def\labelenumi{\alph{enumi}.}
\setcounter{enumi}{2}
\tightlist
\item
  What proportion of \emph{Star Trek: The Original Series} episodes pass
  the Bechdel-Wallace Test? If I select 10 episodes of \emph{Star Trek:
  The Original Series} at random, what is the probability that I will
  see 2 or fewer episodes that pass the Bechdel-Wallace Test?
\end{enumerate}

\textbf{Show your code here:}

\begin{Shaded}
\begin{Highlighting}[]
\CommentTok{##  The library MXB107 should be already loaded, if not type:}

\CommentTok{##  library(MXB107)}

\CommentTok{##  If after loading the library if the dataset episodes not available, type:}

\CommentTok{##  data(episodes)}
\NormalTok{episodes }\OperatorTok\StringTok{ }\KeywordTok{filter}\NormalTok{( Series }\OperatorTok{==}\StringTok{"TOS"}\NormalTok{, Bechdel.Wallace.Test}\OperatorTok{==}\StringTok{"TRUE"}\NormalTok{)}
\end{Highlighting}
\end{Shaded}

\begin{verbatim}
## # A tibble: 5 x 57
##   Series Series.Name Season Episode IMDB.Ranking Title Star.date Air.date
##   <chr>  <chr>        <dbl>   <dbl>        <dbl> <chr> <chr>     <chr>   
## 1 TOS    The Origin~      2       3          7.8 The ~ 3541.9    29/9/67 
## 2 TOS    The Origin~      2       8          7.5 I, M~ 4513.3    3/11/67 
## 3 TOS    The Origin~      3      18          6.3 The ~ 5725.3    31/1/69 
## 4 TOS    The Origin~      3      21          7.2 The ~ 5818.4    28/2/69 
## 5 TOS    The Origin~      3      24          7   Turn~ 5928.5    3/6/69  
## # ... with 49 more variables: Bechdel.Wallace.Test <lgl>, Director <chr>,
## #   Writer.1 <chr>, Writer.2 <chr>, Writer.3 <chr>, Writer.4 <chr>,
## #   Writer.5 <chr>, Writer.6 <chr>, Female.Director <lgl>,
## #   Female.Writer.1 <lgl>, Female.Writer.2 <lgl>, Female.Writer.3 <lgl>,
## #   Female.Writer.4 <lgl>, Female.Writer.5 <lgl>, Female.Writer.6 <lgl>,
## #   Executive.Producer.1 <chr>, Executive.Producer.2 <chr>,
## #   Executive.Producer.3 <chr>, Co.Executive.Producer.1 <chr>,
## #   Co.Executive.Producer.2 <chr>, Co.Executive.Producer.3 <chr>,
## #   Producer.1 <chr>, Producer.2 <chr>, Producer.3 <chr>, Producer.4 <chr>,
## #   Co.Producer.1 <chr>, Co.Producer.2 <chr>, Co.Producer.3 <chr>,
## #   Co.Producer.4 <chr>, Co.Producer.5 <chr>, Associate.Producer.1 <chr>,
## #   Associate.Producer.2 <chr>, Supervising.Producer.1 <chr>,
## #   Supervising.Producer.2 <chr>, Supervising.Producer.3 <chr>,
## #   Co.Supervising.Producer.1 <chr>, Co.Supervising.Producer.2 <chr>,
## #   Line.Producer <chr>, Coordinating.Producer <chr>,
## #   Consulting.Producer.1 <chr>, Consulting.Producer.2 <chr>,
## #   Female.Executive.Producer <lgl>, Female.Co.Executive.Producer <lgl>,
## #   Female.Producer <lgl>, Female.Co.Producer <lgl>,
## #   Female.Associate.Producer <lgl>, Female.Supervising.Producer <lgl>,
## #   Female.Co.Supervising.Producer <lgl>, Female.Line.Producer <lgl>
\end{verbatim}

\begin{Shaded}
\begin{Highlighting}[]
\NormalTok{episodes }\OperatorTok\StringTok{ }\KeywordTok{filter}\NormalTok{( Series }\OperatorTok{==}\StringTok{"TOS"}\NormalTok{)}
\end{Highlighting}
\end{Shaded}

\begin{verbatim}
## # A tibble: 80 x 57
##    Series Series.Name Season Episode IMDB.Ranking Title Star.date Air.date
##    <chr>  <chr>        <dbl>   <dbl>        <dbl> <chr> <chr>     <chr>   
##  1 TOS    The Origin~      1       1          7.3 The ~ 1513.1    8/9/66  
##  2 TOS    The Origin~      1       2          7.2 Char~ 1533.6    15/9/66 
##  3 TOS    The Origin~      1       3          7.8 Wher~ 1312.4    22/9/66 
##  4 TOS    The Origin~      1       4          8   The ~ 1704.2    29/9/66 
##  5 TOS    The Origin~      1       5          7.8 The ~ 1672.1    6/10/66 
##  6 TOS    The Origin~      1       6          6.9 Mudd~ 1329.8    13/10/66
##  7 TOS    The Origin~      1       7          7.6 What~ 2712.4    20/10/66
##  8 TOS    The Origin~      1       8          7.1 Miri  2713.5    27/10/66
##  9 TOS    The Origin~      1       9          7.5 Dagg~ 2715.1    3/11/66 
## 10 TOS    The Origin~      1      10          8.2 The ~ 1512.2    10/11/66
## # ... with 70 more rows, and 49 more variables: Bechdel.Wallace.Test <lgl>,
## #   Director <chr>, Writer.1 <chr>, Writer.2 <chr>, Writer.3 <chr>,
## #   Writer.4 <chr>, Writer.5 <chr>, Writer.6 <chr>, Female.Director <lgl>,
## #   Female.Writer.1 <lgl>, Female.Writer.2 <lgl>, Female.Writer.3 <lgl>,
## #   Female.Writer.4 <lgl>, Female.Writer.5 <lgl>, Female.Writer.6 <lgl>,
## #   Executive.Producer.1 <chr>, Executive.Producer.2 <chr>,
## #   Executive.Producer.3 <chr>, Co.Executive.Producer.1 <chr>,
## #   Co.Executive.Producer.2 <chr>, Co.Executive.Producer.3 <chr>,
## #   Producer.1 <chr>, Producer.2 <chr>, Producer.3 <chr>, Producer.4 <chr>,
## #   Co.Producer.1 <chr>, Co.Producer.2 <chr>, Co.Producer.3 <chr>,
## #   Co.Producer.4 <chr>, Co.Producer.5 <chr>, Associate.Producer.1 <chr>,
## #   Associate.Producer.2 <chr>, Supervising.Producer.1 <chr>,
## #   Supervising.Producer.2 <chr>, Supervising.Producer.3 <chr>,
## #   Co.Supervising.Producer.1 <chr>, Co.Supervising.Producer.2 <chr>,
## #   Line.Producer <chr>, Coordinating.Producer <chr>,
## #   Consulting.Producer.1 <chr>, Consulting.Producer.2 <chr>,
## #   Female.Executive.Producer <lgl>, Female.Co.Executive.Producer <lgl>,
## #   Female.Producer <lgl>, Female.Co.Producer <lgl>,
## #   Female.Associate.Producer <lgl>, Female.Supervising.Producer <lgl>,
## #   Female.Co.Supervising.Producer <lgl>, Female.Line.Producer <lgl>
\end{verbatim}

\begin{Shaded}
\begin{Highlighting}[]
\NormalTok{a =}\StringTok{ }\DecValTok{5}\OperatorTok{/}\DecValTok{80}
\NormalTok{a}
\end{Highlighting}
\end{Shaded}

\begin{verbatim}
## [1] 0.0625
\end{verbatim}

\begin{Shaded}
\begin{Highlighting}[]
\KeywordTok{sum}\NormalTok{(}\KeywordTok{dbinom}\NormalTok{(}\DecValTok{0}\OperatorTok{:}\DecValTok{2}\NormalTok{, }\DecValTok{10}\NormalTok{, a))}
\end{Highlighting}
\end{Shaded}

\begin{verbatim}
## [1] 0.9789929
\end{verbatim}

\textbf{Type your answer here:} 0.0625 passed the test. The second
question can be answered by using binomial probability distribution,
where X\textasciitilde Binom(0 to 2, 10 episodes and the chance of pass
the test). Which the answer is 0.9789929

\begin{enumerate}
\def\labelenumi{\alph{enumi}.}
\setcounter{enumi}{3}
\tightlist
\item
  Now assume that I sample episodes at random from all 704 episodes of
  \emph{Star Trek} and the proportion of all episodes that pass the
  Bechdel-Wallace Test is \(0.52\). If I select 100 episodes at random
  from all the episodes of \emph{Star Trek} what is probability that I
  see less than 50 episodes that pass the Bechdel-Wallace Test. Compute
  this using the binomial probability distribution, the Poisson
  probability distribution, and the Gaussian distribution. Compare and
  contrast the results.
\end{enumerate}

\textbf{Show your code here:}

\begin{Shaded}
\begin{Highlighting}[]
\CommentTok{##  The library MXB107 should be already loaded, if not type:}

\CommentTok{##  library(MXB107)}

\CommentTok{##  If after loading the library if the dataset episodes not available, type:}

\CommentTok{##  data(episodes)}
\KeywordTok{sum}\NormalTok{(}\KeywordTok{dbinom}\NormalTok{(}\DecValTok{0}\OperatorTok{:}\DecValTok{49}\NormalTok{, }\DecValTok{100}\NormalTok{, }\FloatTok{0.52}\NormalTok{))}
\end{Highlighting}
\end{Shaded}

\begin{verbatim}
## [1] 0.3081545
\end{verbatim}

\begin{Shaded}
\begin{Highlighting}[]
\NormalTok{l <-}\StringTok{ }\DecValTok{100}\OperatorTok{*}\FloatTok{0.52}
\KeywordTok{sum}\NormalTok{(}\KeywordTok{ppois}\NormalTok{(}\DecValTok{49}\NormalTok{,}\DecValTok{52}\NormalTok{))}
\end{Highlighting}
\end{Shaded}

\begin{verbatim}
## [1] 0.3721497
\end{verbatim}

\begin{Shaded}
\begin{Highlighting}[]
\KeywordTok{pnorm}\NormalTok{(}\DecValTok{49}\NormalTok{, }\DecValTok{52}\NormalTok{, }\DecValTok{5}\NormalTok{)}
\end{Highlighting}
\end{Shaded}

\begin{verbatim}
## [1] 0.2742531
\end{verbatim}

\textbf{Type your answer here:} The correct answer to this question is
using binomial probability distribution. Poisson's lamda is E, where E =
np. The number 52 in gussian is 52, which is E, and E = np. The standard
deviation is from binomail's standard deviation, which is equal to
variance, where V = np*1-p.~standard deviation is the squar root of
variance.

The answer is Bin = 0.3081545 Pos = 0.3721497 Gussian = 0.2742531

\hypertarget{question-3-2}{%
\subsubsection{Question 3}\label{question-3-2}}

\begin{enumerate}
\def\labelenumi{\alph{enumi}.}
\tightlist
\item
  Show that as \(n\rightarrow \infty\) and \(p\rightarrow 0\) the
  probability distribution of a random variable \(X\sim Binom(n,p)\)
  converges to a Poisson probability distribution.
\end{enumerate}

\textbf{Type your answer here:}

let X be a random variable, X follows a binomial probability
distribution. If define \(\lambda = np\) then the limit as \$~n
\rightarrow \infty \$ and \(\ p \rightarrow 0\) is the Poisson
probability distribution wirh rate \(\lambda\). the derivation will be

\[
p(x) = \frac{n!}{(n-x)!x!}\binom{ \lambda }{n}(1-\frac{\lambda}{n})^{n-x}
\] \[
= \frac{\lambda^x}{x!}\frac{n!}{(n-x)!}\frac{1}{n^x}(1-\frac{\lambda}{n})^{n}(1-\frac{\lambda}{n})^{-x}
\] Since \$~n \rightarrow \infty \$ \[
(1-\frac{\lambda}{n})^n \rightarrow e^{-\lambda} 
\] the term \[
\frac{\lambda^x}{x!}
\] is constant, and the remaining tems go to 1 so \[
p(x) = \frac{\lambda^xe^{-\lambda}}{x!}
\] The mean and the vairnce of X are \[
E(X) = \lambda,
Var(X) = \lambda
\]

\begin{enumerate}
\def\labelenumi{\alph{enumi}.}
\setcounter{enumi}{1}
\tightlist
\item
  For \emph{Star Trek: The Original Series} plot the probability
  distribution for the number of episodes out ten that that would pass
  the Bechdel-Wallace Test. Use the Binomial and the Poisson
  distributions. Compare and discuss the results.
\end{enumerate}

\textbf{Show your code here:}

\begin{Shaded}
\begin{Highlighting}[]
\CommentTok{##  The library MXB107 should be already loaded, if not type:}

\CommentTok{##  library(MXB107)}

\CommentTok{##  If after loading the library if the dataset episodes not available, type:}

\CommentTok{##  data(episodes)}


\NormalTok{x <-}\StringTok{ }\KeywordTok{seq}\NormalTok{(}\DecValTok{0}\OperatorTok{:}\DecValTok{10}\NormalTok{)}
\NormalTok{f_binom<-}\KeywordTok{dbinom}\NormalTok{(x,}\DataTypeTok{size =}\DecValTok{10}\NormalTok{, }\DataTypeTok{prob =} \FloatTok{0.5}\NormalTok{)}
\NormalTok{f_pois <-}\StringTok{ }\KeywordTok{dpois}\NormalTok{(x,}\DataTypeTok{lambda =} \DecValTok{5}\NormalTok{)}
\NormalTok{df <-}\StringTok{ }\KeywordTok{tibble}\NormalTok{(}\DataTypeTok{x =} \KeywordTok{rep}\NormalTok{(x,}\DecValTok{2}\NormalTok{),}\DataTypeTok{y =} \KeywordTok{c}\NormalTok{(f_binom,f_pois),}
             \DataTypeTok{distribution =} \KeywordTok{c}\NormalTok{(}\KeywordTok{rep}\NormalTok{(}\StringTok{"binomial"}\NormalTok{,}\KeywordTok{length}\NormalTok{(f_binom)),}\KeywordTok{rep}\NormalTok{(}\StringTok{"Poisson"}\NormalTok{,}\KeywordTok{length}\NormalTok{(f_pois))))}
\KeywordTok{ggplot}\NormalTok{(df,}\KeywordTok{aes}\NormalTok{(}\DataTypeTok{x =}\NormalTok{ x,}\DataTypeTok{y =}\NormalTok{ y))}\OperatorTok{+}
\StringTok{  }\KeywordTok{geom_step}\NormalTok{(}\KeywordTok{aes}\NormalTok{(}\DataTypeTok{color =}\NormalTok{ distribution))}
\end{Highlighting}
\end{Shaded}

\begin{center}\includegraphics{MXB107-Assessment-01_final_final_files/figure-latex/unnamed-chunk-16-1} \end{center}

\textbf{Type your answer here:} Both results would get more about the
same as n leads to infinite and p to zero. This question n is 10 which
is very small compare to infiinite, which lead to a different result as
binomial. The reason why if n is inifinite the result will be the same
has already been answered on above question.

\begin{enumerate}
\def\labelenumi{\alph{enumi}.}
\setcounter{enumi}{2}
\tightlist
\item
  What is the relationship between the Poisson and Exponential
  probability distributions?
\end{enumerate}

\textbf{Type your answer here:} Poisson distribution is used to deal
with number of occurrences in a fixed amount of time. On the other hand,
exponential distribution is used in the time between occurrences of
successive events as time flow by continuously.

Assume that the average episode is 45 minutes long, and given the
probability that a given episode has a probability of passing the
Bechdel-Wallace Test of \(p=0.52\), that is the equivalent \(0.693\)
instances of passing the Bechdel-Wallace Test per hour of \emph{Star
Trek} viewing.

\begin{enumerate}
\def\labelenumi{\alph{enumi}.}
\setcounter{enumi}{2}
\tightlist
\item
  If I watch ten hours of \emph{Star Trek} (assume the hours are
  completely random), what is the probability that I see more than 7
  instances of passing the Bechdel-Wallace Test.
\end{enumerate}

\begin{Shaded}
\begin{Highlighting}[]
\KeywordTok{ppois}\NormalTok{(}\DecValTok{7}\NormalTok{, }\FloatTok{6.93}\NormalTok{,}\DataTypeTok{lower=}\OtherTok{FALSE}\NormalTok{)}
\end{Highlighting}
\end{Shaded}

\begin{verbatim}
## [1] 0.3908572
\end{verbatim}

\textbf{Type your answer here:}

we can apply poisson probability distribution here, where n = 7 and P is
6.93 . Answer is 0.3908572

\begin{enumerate}
\def\labelenumi{\alph{enumi}.}
\setcounter{enumi}{3}
\tightlist
\item
  What is the probability that I will have to watch more than three
  hours to see one instance of passing the Bechdel-Wallace Test
\end{enumerate}

\begin{Shaded}
\begin{Highlighting}[]
\KeywordTok{pexp}\NormalTok{(}\DecValTok{3}\NormalTok{,}\DecValTok{1}\OperatorTok{/}\FloatTok{0.693}\NormalTok{,}\DataTypeTok{lower.tail =} \OtherTok{FALSE}\NormalTok{)}
\end{Highlighting}
\end{Shaded}

\begin{verbatim}
## [1] 0.01318066
\end{verbatim}

\textbf{Type your answer here:}

answer 0.1250552

\hypertarget{question-4-1}{%
\subsubsection{Question 4}\label{question-4-1}}

\begin{enumerate}
\def\labelenumi{\alph{enumi}.}
\tightlist
\item
  Define the \(Z\)-score, or how we convert a Gaussian random variable
  to a Standard Gaussian random variable.
\end{enumerate}

\textbf{Type your answer here:}

For \(X\sim N(\mu,\sigma^2)\), \[
Z = Y−μ/σ∼N(0,1).
\] where \(Z\sim N(0,1)\).

if\(E(Y) = \mu\) and \(Var(Y) = σ^2\) then \[
E(Y) - \mu = \mu - \mu
\] which equal to 0, and \[
Var(Y - \mu) = σ^2
\] \[
Var(\frac{Y-\mu}{σ}) = \frac{1}{σ^2}Var(Y-\mu)
\] which equal to 1, which means \[
Z = \frac{Y-\mu}{σ}\sim N(0,1)
\]

\begin{enumerate}
\def\labelenumi{\alph{enumi}.}
\setcounter{enumi}{1}
\tightlist
\item
  For \(X\sim N(4.3,2.7)\) find \(Pr(X>5)\)
\end{enumerate}

\begin{Shaded}
\begin{Highlighting}[]
\KeywordTok{pnorm}\NormalTok{(}\DecValTok{5}\NormalTok{, }\DataTypeTok{mean=}\FloatTok{4.3}\NormalTok{, }\DataTypeTok{sd=}\KeywordTok{sqrt}\NormalTok{(}\FloatTok{2.7}\NormalTok{), }\DataTypeTok{lower.tail=}\OtherTok{FALSE}\NormalTok{)}
\end{Highlighting}
\end{Shaded}

\begin{verbatim}
## [1] 0.3350516
\end{verbatim}

\textbf{Type your answer here:} 0.3350516

\begin{enumerate}
\def\labelenumi{\alph{enumi}.}
\setcounter{enumi}{2}
\tightlist
\item
  Assume that the IMDB rankings for episodes of \emph{Star Trek} follow
  a Gaussian distribution with \(\mu = 7.55\) and \(\sigma^2=0.60\)
  based on the Gaussian distribution, what is the probability that a
  randomly selected episode will have an IMDB ranking of less than 7?
\end{enumerate}

\textbf{Type your answer here:}

\begin{Shaded}
\begin{Highlighting}[]
\DecValTok{1} \OperatorTok{-}\StringTok{ }\KeywordTok{pnorm}\NormalTok{(}\DecValTok{7}\NormalTok{, }\DataTypeTok{mean=}\FloatTok{7.55}\NormalTok{, }\DataTypeTok{sd=}\KeywordTok{sqrt}\NormalTok{(}\FloatTok{0.6}\NormalTok{), }\DataTypeTok{lower.tail=}\OtherTok{FALSE}\NormalTok{)}
\end{Highlighting}
\end{Shaded}

\begin{verbatim}
## [1] 0.2388375
\end{verbatim}

answer is 0.2388375

\begin{enumerate}
\def\labelenumi{\alph{enumi}.}
\setcounter{enumi}{3}
\tightlist
\item
  Assume that the IMDB rankings for episodes of \emph{Star Trek} follow
  a Gaussian distribution with \(\mu = 7.55\) and \(\sigma^2=0.60\)
  based on the Gaussian distribution, what proportion of epsiodes have
  an IMDB ranking of over 7.9? What is the actual proportion of episodes
  with an IMDB ranking of over 7.9? Compare your results.
\end{enumerate}

\textbf{Type your answer here:}

\begin{Shaded}
\begin{Highlighting}[]
\KeywordTok{pnorm}\NormalTok{(}\FloatTok{7.9}\NormalTok{, }\DataTypeTok{mean=}\FloatTok{7.55}\NormalTok{, }\DataTypeTok{sd=}\KeywordTok{sqrt}\NormalTok{(}\FloatTok{0.6}\NormalTok{), }\DataTypeTok{lower.tail=}\OtherTok{FALSE}\NormalTok{)}
\end{Highlighting}
\end{Shaded}

\begin{verbatim}
## [1] 0.3256892
\end{verbatim}

\begin{Shaded}
\begin{Highlighting}[]
\NormalTok{episodes }\OperatorTok\StringTok{ }\KeywordTok{filter}\NormalTok{(IMDB.Ranking }\OperatorTok{>}\StringTok{ }\FloatTok{7.9}\NormalTok{)}
\end{Highlighting}
\end{Shaded}

\begin{verbatim}
## # A tibble: 216 x 57
##    Series Series.Name Season Episode IMDB.Ranking Title Star.date Air.date
##    <chr>  <chr>        <dbl>   <dbl>        <dbl> <chr> <chr>     <chr>   
##  1 TOS    The Origin~      1       4          8   The ~ 1704.2    29/9/66 
##  2 TOS    The Origin~      1      10          8.2 The ~ 1512.2    10/11/66
##  3 TOS    The Origin~      1      11          8.4 The ~ 3012.4    17/11/66
##  4 TOS    The Origin~      1      12          8.3 The ~ 3013.1    24/11/66
##  5 TOS    The Origin~      1      14          9   Bala~ 1709.2    15/12/66
##  6 TOS    The Origin~      1      18          8.1 Arena 3045.6    19/1/67 
##  7 TOS    The Origin~      1      19          8   Tomo~ 3113.2    26/1/67 
##  8 TOS    The Origin~      1      22          8.9 Spac~ 3141.9    16/2/67 
##  9 TOS    The Origin~      1      23          8.2 A Ta~ 3192.1    23/2/67 
## 10 TOS    The Origin~      1      24          8   This~ 3417.3    2/3/67  
## # ... with 206 more rows, and 49 more variables: Bechdel.Wallace.Test <lgl>,
## #   Director <chr>, Writer.1 <chr>, Writer.2 <chr>, Writer.3 <chr>,
## #   Writer.4 <chr>, Writer.5 <chr>, Writer.6 <chr>, Female.Director <lgl>,
## #   Female.Writer.1 <lgl>, Female.Writer.2 <lgl>, Female.Writer.3 <lgl>,
## #   Female.Writer.4 <lgl>, Female.Writer.5 <lgl>, Female.Writer.6 <lgl>,
## #   Executive.Producer.1 <chr>, Executive.Producer.2 <chr>,
## #   Executive.Producer.3 <chr>, Co.Executive.Producer.1 <chr>,
## #   Co.Executive.Producer.2 <chr>, Co.Executive.Producer.3 <chr>,
## #   Producer.1 <chr>, Producer.2 <chr>, Producer.3 <chr>, Producer.4 <chr>,
## #   Co.Producer.1 <chr>, Co.Producer.2 <chr>, Co.Producer.3 <chr>,
## #   Co.Producer.4 <chr>, Co.Producer.5 <chr>, Associate.Producer.1 <chr>,
## #   Associate.Producer.2 <chr>, Supervising.Producer.1 <chr>,
## #   Supervising.Producer.2 <chr>, Supervising.Producer.3 <chr>,
## #   Co.Supervising.Producer.1 <chr>, Co.Supervising.Producer.2 <chr>,
## #   Line.Producer <chr>, Coordinating.Producer <chr>,
## #   Consulting.Producer.1 <chr>, Consulting.Producer.2 <chr>,
## #   Female.Executive.Producer <lgl>, Female.Co.Executive.Producer <lgl>,
## #   Female.Producer <lgl>, Female.Co.Producer <lgl>,
## #   Female.Associate.Producer <lgl>, Female.Supervising.Producer <lgl>,
## #   Female.Co.Supervising.Producer <lgl>, Female.Line.Producer <lgl>
\end{verbatim}

\begin{Shaded}
\begin{Highlighting}[]
\DecValTok{216}\OperatorTok{/}\DecValTok{704}
\end{Highlighting}
\end{Shaded}

\begin{verbatim}
## [1] 0.3068182
\end{verbatim}

answer is 0.3256892 actual is 0.3068182

\end{document}
